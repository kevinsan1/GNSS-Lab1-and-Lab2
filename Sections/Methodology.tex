\section{Methodology} 

% (fold)
\label{sec:methodology}

% Describe the methods and instruments used to solve the problem. 
Matlab was used to solve the problem and used the GPS single point positioning algorithm.\cite{milanGPS} 

The steps from this algorithm are documented in the Matlab code, Section~\ref{sec:main_code}. The pseudo ranges and navigation messages are used to calculate a more precise position of the receiver.  The shift in position is a measure of the time for the signal to travel from satellite to receiver. 

This time traveled by the signal multiplied by the speed of light is the pseudo range.  

All the satellites were used in this calculation in order to get the best possible approximation of the receiver coordinates.  Since we will be using least-squares method to solve for the parameters, the more independent equations we have, the better the approximation will be.  

First, the signal propagation time was calculated using,
\begin{equation*}
	\Delta t_A^s = P_A^s(\tilde{t_A})/c
\end{equation*}
which was then used to calculate the nominal transmission time,
\begin{equation*}
	\tilde{t}^s = \tilde{t_A} - P_A^s(\tilde{t_A})/c
\end{equation*}
where $\tilde{t_A}$ is the nominal time of signal reception by the receiver.  For this lab report, $\tilde{t_A}=90840~$seconds, or 2 days, 1 hour, and 14 minutes.  The weeks in the year was omitted because the difference between epoch times was less than a day and therefore would have cancelled out.

Next the satellite clock correction, $\delta t_{L1}^s$, was computed using equations,
\begin{align*}
	\delta t_{L1}^s &= \Delta t_{SV}-T_{GD} \\
	\Delta t_{SV} &= a_{f0} + a_{f1}(\tilde{t}^s - t_{oc}) + a_{f2}(\tilde{t}^s - t_{oc})^2 + \Delta t_r
\end{align*}
but the relativistic correction, $\Delta t_r = Fe\sqrt{A} \sin(E_k)$ was neglected until the eccentric anomaly and satellite coordinates were computed.  Next, the system time of signal transmission, $t^s$ was calculated using the satellite clock correction. The eccentric anomaly was calculated using Kepler's equation,
\begin{equation*}
	E_k = M_k + e \sin(E_k)
\end{equation*}
 and the solution was iterated until convergence.  The relativistic correction is now calculated and $t^s$ is corrected in order to compute the satellite coordinates at that time given by the equations,
 \begin{align}
 	X^s &=x'_k \cos \Omega_k -y' \cos i \sin \Omega_k \nonumber \\
	Y^s &=x'_k \sin \Omega_k +y' \cos i \cos \Omega_k \nonumber \\
	Z^s &= y' \sin i_k  \nonumber
 \end{align}
 Now that the satellite coordinates have been calculated, the satellite clock correction is corrected again to include the relativistic correction, $\Delta t_r$.  Now the approximate distance of the satellite is calculated using the equation,
 \begin{equation*}
\rho _{\text{A0}}^s\left(\tilde{t}_A\right) = 
\sqrt{\left(
X^s-X_{\text{A0}}+\dot{\Omega}_eY_{\text{A0}}  \text{$\Delta $t}^s
\right)^2+\left(
Y^s-Y_{\text{A0}}-\dot{\Omega}_eX_{\text{A0}} \text{$\Delta $t}^s
\right)^2+\left(
Z^s-Z_{\text{A0}}
\right)^2}
 \end{equation*}
 
 The calculation of these parameters is done for all satellites we are interested in.  Since we are computing four unknown parameters, the satellite lower limit is also four.  More than four should only increase our accuracy assuming they aren't outliers. The result will be used to get more accurate coordinates for the receiver.  
% section methodology (end)
