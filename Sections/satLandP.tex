\section{Matlab Function: satLandP.m} % (fold)
\label{sec:matlab_function_satlandp_m}

% section matlab_function_satlandp_m (end)
\subsection*{Contents}

\begin{itemize}
\setlength{\itemsep}{-1ex}
   \item Constants
   \item Importing numbers from table 3 for a satellite
   \item 1. Compute signal propagation time by (13)
   \item 2. Compute signal transmission time by (14)
   \item 3. Compute satellite clock correction dtsL1
   \item 4. Compute ts using the correction from the step 3.
   \item 5. Compute eccentric anomaly (Table 2)
   \item 6. Compute dtr by (26) and ts by (15).
   \item 7. Compute satellite coordinates Xs, Ys, Zs, for time ts
   \item 8. Compute satellite clock correction dtsL1 by (24) - (27)
   \item 9. Compute tropospheric correction T\_A\_to\_s (tA)
   \item 10. Compute ionospheric correction I\_A\_to\_s (tA)
   \item 11. Compute approximate distance rho\_A0\_to\_s (tA) by (11).
   \item 12. Repeat steps 1 - 11 for all measured satellites.
   \item 13. Compute elements of vector L (19).
   \item 14. Compute elements of matrix A (20); a\_x\_to\_s , a\_y\_to\_s , a\_z\_to\_s by (12)
\end{itemize}
\begin{verbatim}
function [ Lmatrix, Amatrix,rho_f,Xs_f,Ys_f,Zs_f,P1_f,dtsL1_dtr_f,tAtoS_f]...
    = satLandP( 
	satelliteNumberOrder,P1_f,navfiles,XA0,YA0,ZA0,ndaysf,nhoursf,nminutesf,nsec
	ondsf)
\end{verbatim}


\subsection*{Constants}

\begin{verbatim}
c = 299792458; % speed of light (m/s)
mu = 3.986005e14; % universal gravitational parameter (m/s)^3
omega_e_dot = 7.2921151467e-5; % earth rotation rate (rad/s)
F = -4.442807633e-10; % s/m^1/2
\end{verbatim}


\subsection*{Importing numbers from table 3 for a satellite}

\begin{verbatim}
i = 1:8:112;
sat = navfiles(i(satelliteNumberOrder):i(satelliteNumberOrder)+8,:);
sat = transpose(cell2mat(sat));
sat = num2cell(sat);
% Imports numbers to all variables
[~,     af0,        af1,        af2,...
    ~,       crs,        change_n,         m0,...
    cuc,        ec,         cus,        sqrtA,...
    toe,        cic,        omega0,     cis,...
    i0,         crc,        w,          omegadot,...
    idot,       ~,          ~,          ~,...
    ~,          ~,          tgd,        ~,...
    ~]...
    =sat{:};
\end{verbatim}


\subsection*{1. Compute signal propagation time by (13)}

\begin{verbatim}
tA_nom = seconds_in_week(ndaysf,nhoursf,nminutesf,nsecondsf); % 2 days, 1 hour,
% 14 minutes My Time
tAtoS_f = P1_f/c; % signal propagation time
\end{verbatim}


\subsection*{2. Compute signal transmission time by (14)}

\begin{verbatim}
tS_nom = tA_nom - P1_f/c;
\end{verbatim}


\subsection*{3. Compute satellite clock correction dtsL1}

\begin{par}
by (24) and (25), neglect dtr
\end{par} \vspace{1em}
\begin{verbatim}
t_oc = toe; % I believe this is true, but not sure
change_tsv_f = af0 + af1*(tS_nom-t_oc)+af2*(tS_nom-t_oc)^2; % (25)
dtsL1 = change_tsv_f - tgd; % (24)
\end{verbatim}


\subsection*{4. Compute ts using the correction from the step 3.}

\begin{verbatim}
ts_f = tS_nom - dtsL1;
\end{verbatim}


\subsection*{5. Compute eccentric anomaly (Table 2)}

\begin{par}
ek = mk + ec*sin(ek)
\end{par} \vspace{1em}
\begin{verbatim}
A = sqrtA^2;
n0 = sqrt(mu/A^3); % Computed mean motion
n = n0 + change_n;
tk = ts_f - toe;
tk = fixTk(tk); % if,then for table 2 of tk
mk = m0 + n*tk;
Ek = keplersEquation(mk,ec);
\end{verbatim}


\subsection*{6. Compute dtr by (26) and ts by (15).}

\begin{verbatim}
change_tr = F*ec*sqrtA*sin(Ek); %(26)
ts_with_dtr = ts_f - change_tr;
\end{verbatim}


\subsection*{7. Compute satellite coordinates Xs, Ys, Zs, for time ts}

\begin{par}
- Table 2 Calculate rk
\end{par} \vspace{1em}
\begin{verbatim}
vk = atan2((sqrt(1-ec^2)*sin(Ek)/(1-ec*cos(Ek))),...
    ((cos(Ek)-ec)/(1-ec*cos(Ek))));
Phik = vk + w;
drk = crs*sin(2*Phik) + crc*cos(2*Phik);
rk = A*(1-ec*cos(Ek)) + drk;  % Corrected radius
% Calculate uk
duk = cus*sin(2*Phik) + cuc*cos(2*Phik);
uk = Phik + duk;
% Calculate ik
dik = cis*sin(2*Phik) + cic*cos(2*Phik);
ik = i0 + dik + idot*tk;
% Calculate omega's
omegak = omega0 + (omegadot-omega_e_dot)*tk - omega_e_dot*toe;
% Calculate xkp and ykp
xkp = rk*cos(uk);
ykp = rk*sin(uk);
% Calculate xk,yk,zk -> Xs, Ys, Zs for time ts
xk = xkp*cos(omegak) - ykp*cos(ik)*sin(omegak);
yk = xkp*sin(omegak) + ykp*cos(ik)*cos(omegak);
zk = ykp*sin(ik);
Xs_f = xk;
Ys_f = yk;
Zs_f = zk;
\end{verbatim}


\subsection*{8. Compute satellite clock correction dtsL1 by (24) - (27)}

\begin{verbatim}
dtsL1_dtr_f = change_tsv_f + change_tr - tgd; % (24)
\end{verbatim}


\subsection*{9. Compute tropospheric correction T\_A\_to\_s (tA)}



\subsection*{10. Compute ionospheric correction I\_A\_to\_s (tA)}



\subsection*{11. Compute approximate distance rho\_A0\_to\_s (tA) by (11).}

\begin{verbatim}
rho_f = sqrt(...
    (Xs_f - XA0 + omega_e_dot * YA0 * tAtoS_f)^2 + ... % x^2
    (Ys_f - YA0 - omega_e_dot * XA0 * tAtoS_f)^2 + ... % y^2
    (Zs_f - ZA0)^2   ... % z^2
    );
% dtA = 0;
% rho_A_to_s = P1 + c*dtsL1_with_dtr - c*dtA; % (8) dtA =\= 0
\end{verbatim}


\subsection*{12. Repeat steps 1 - 11 for all measured satellites.}



\subsection*{13. Compute elements of vector L (19).}

\begin{verbatim}
Lmatrix = P1_f - rho_f + c*dtsL1_dtr_f;
\end{verbatim}


\subsection*{14. Compute elements of matrix A (20); a\_x\_to\_s , a\_y\_to\_s , a\_z\_to\_s by (12)}

\begin{verbatim}
Amatrix = 1/rho_f*[-(Xs_f - XA0),-(Ys_f - YA0),-(Zs_f - ZA0),rho_f];
\end{verbatim}
\begin{verbatim}
end
\end{verbatim}

    
